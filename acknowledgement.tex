\begin{acknowledgments}
Era da tempo che aspettavo questo momento. Dal primo giorno di universitá. 8 lunghi, interminabili anni (me la sono presa abbastanza comoda) a fare finta di studiare con l'obbiettivo, un giorno, di poter finalmente far finta di lavorare. Un ottimo esempio di coerenza. Arrivato a questo punto, peró, c'é un'ultima convenzione sociale a cui devo adempiere: i ringraziamenti. Non sono solito dire "grazie", ma é un occasione speciale e ci sono alcune persone a cui lo devo.\newline
Ringrazio i miei genitori, gli azionisti di maggioranza di questo progetto: senza di loro mi sarei dovuto pagare le tasse da solo e hai voglia a consegnare pizze per Deliveroo! Scherzo, grazie di tutto Mamma e Papá, soprattutto per avermi incoraggiato a proseguire questa impresa.\newline
Ringrazio (in ordine alfabetico per non creare faide interne) David, Francesco, Gianmarco, Giordano e Marco. In una parola, ringrazio La Compagnia che, scusate il gioco di parole, mi accompagna da oltre 20 anni e che spero faccia per tanto tempo ancora.\newline
Ringrazio Alessandro e Manuel, compagni di viaggio, suppur a distanza nell'ultimo periodo, dal primo anno di questa odissea chiamata "Corso di Laurea in Ingegneria Informatica".\newline
Ringrazio tutti i miei, ormai, ex-colleghi incontrati durante gli anni, quelli veri, quelli che non si fanno problemi a condividere (e a volte a farti copiare). Per gli altri, senza rancore eh, auguro la piú grande delle maledizioni: qualche anno di fuoricorso.\newline
Ringrazio la pandemia e la DaD, che seppur rovinandomi l'Erasmus mi hanno dato una "spinta" negli ultimi esami. Beccarsi il Covid a Bratislava é stato un piccolo prezzo che sono stato felice di pagare.\newline
Ringrazio la mia relatrice, la prof. Silvia Bonomi, che ha assecondato la mia folle scelta di scrivere questa tesi, con lo stesso stile con cui affronto qualunque problema: scherzandoci sú e senza prendermi sul serio.\newline
Dedico tutto questo a nonno Toni, il mio fan nº 1, che ha sempre tifato per me e che sicuramente non si sará perso questo mio traguardo. E dal quale ho ereditato la mia "pragmaticitá": fosse stato qui avrebbe cercato di convicermi che, in un modo o nell'altro, due chiodi e una martellata sarebbero state delle ottime soluzioni nel campo della cybersecurity.\newline
Dedico tutto questo alla mia famiglia, ai miei parenti vicini e lontani, ma soprattutto a me stesso, che contro ogni pronostico sono diventato il primo ingegnere della Compagnia. E tutto sommato, non é stato neanche troppo complicato.\newline
Scusate la brevitá, ma ho dato fondo alle mie ultime abilitá di scrittura per stendere questo papiro che sará stato letto da una persona soltanto, me compreso.\vspace{1cm}\newline
P.S. Il prossimo momento di espansivitá a data da destinarsi.

\end{acknowledgments}