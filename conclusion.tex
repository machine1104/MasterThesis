\chapter{Conclusions}
The main purpose of this paper is to present a high-level overview of the situation regarding risk management within a critical infrastructure in the form of a simple illustrated guide (sometimes it is enough to look at the figures). The basic concepts of cyber risk management were analyzed in such a way that the guide was also accessible to non-experts in the cybersecurity subject (from Dummy to Dummies). The general criticisms that an organization that manages information must face and the tools that it can use to defend itself were exposed. Some examples of cyber risk management frameworks have been reported and each has been analyzed in detail to understand the subtle differences between them. A comparison was made between the different methodologies and frameworks that cybersecurity managers can use, from the most complex and specific to the most adaptable to their needs. Finally, the case of healthcare infrastructure was examined as an example of critical infrastructure. What emerged from the sources examined is that healthcare is not yet totally self-sufficient and mature on the head of cybersecurity. Few resources are invested in cyber risk management even if the entire infrastructure is now based on smart devices. MCPSs are subject to threats typical of the IoT world and their modularity increases the chances of being attacked (larger attack surface, remember?).
The main target of these attacks is the personal data managed by the healthcare infrastructure which is now worth much more than stolen credit card data.
The main cause of criticism appears to be the lack of cyber training and awareness of the personnel within the infrastructure. This guide can be used as an introduction to cyber risk management or, if printed, it can be use to balance your shaky table.