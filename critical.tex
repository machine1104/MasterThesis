\chapter{Healthcare: a Critical Infrastructure}
We have finally analyzed most of the options available to prevent the world from collapsing on us after one of our employees opens a "harmless" email at work. This was the theory though. What happens in practice? To understand the real situation of a cyber soldier intent on managing the fragility of a critical infrastructure, we will take healthcare infrastructure as an example. But before that, let's make it clear what the hell a critical infrastructure is.
\definition{Critical Infrastructure}{\begin{center}"System and assets, whether physical or virtual, so vital to the U.S. that the incapacity or destruction of such systems and assets would have a debilitating impact on security, national economic security, national public health or safety, or any combination of those matters." \cite{Paulsen2019}\medskip\end{center} A critical infrastructure is a backbone of the state in which it resides. Without it, the state would collapse. Imagine a country without hospitals. You'll end up getting treated by the butcher.}The concept obviously isn't limited to the U.S. and healthcare is not the only critical infrastructure. Other examples are the vast network of highways, connecting bridges and tunnels, railways, utilities and buildings necessary to maintain normalcy in daily life. Transportation, commerce, clean water and electricity all rely on these vital systems. Each of them is now based on ICT and therefore subject to new and continuous risks. The blackout of the entire traffic light system alone can generate chaos in any metropolis in the world. You may be late for work, but someone else may lose their life in an ambulance that doesn't get to the hospital on time. If the city's water system stopped, no one would have drinking water anymore. This would lead to an assault on supermarkets to grab the last supplies of water at crazy prices, with brawls and looting worthy of the best Black Fridays. There is no need for me to explain to you what could happen if an attacker took control of all railroad interchanges. Each infrastructure has its own peculiarities and the precautions taken can vary.
\section{MCPS}
Let's now go into the details of the healthcare infrastructure. The days when the village doctor was clad in colorful feathers and wearing bone necklaces are long gone. We are in the 2020s and our doctors rely on technology, as well as medicine. \cite{Nair2019} and \cite{Dey2018} perfectly explain what this type of infrastructure is characterized by. Many countries suffer from a shortage of healthcare personnel which decreases the quality of the service offered and increases its costs. The use of sensors and smart devices has made it possible to solve this problem, relieving the workload of the staff, but has led to problems of inefficiency in processing the large amount of data that has arisen. For every problem there is a solution and in this case it is the creation of specific frameworks for what we now call \textit{Medical Cyber-Physical System (MCPS)}. The devices used in a hospital follow the pace of technological progress and are increasingly interconnected in complex configurations to form systems of systems that extent from tiny to ultra-large geographically distributed electronic records systems using networking. Pace makers, medical ventilators, infusion pumps, but also simpler things such as glucose or cholesterol sensors are now supported by computer systems that bring advantages , but also some intrinsic vulnerabilities such as possibility that they will be manipulated in a fraudulent manner. Patient data are no longer handwritten on a large dusty book, but are entered and cataloged in electronic databases, which can be consulted by experts to make remote diagnoses or to perform big data analyzes. However, these data must be protected from access by unauthorized persons so that the privacy of patients is respected also in compliance with the \textit{Health Insurance Portability and Accountability Act (HIPAA)} and the aforementioned GDPR. MCPSs include human interaction as central aspect and therefore the human element must be rigorously analyzed, understood and modeled.
\section{Attack analysis}
Relying on cyberspace improves the quality of the service but also increases the attack surface. A MCPS, a smart hospital or in any other way you want to call it is nothing more than an IoT system. Certainly more complex than the Google Home you use to turn on the lights, but the basic sense is the same. Sensors, networks and data processing. "Same same, but different, but still same". For this reason, healthcare infrastructure is subject to all those threats that are specific to the IoT world. \cite{Djenna2018} show the five main threats source that an IoT-based hospital must face.
\begin{itemize}
    \item \textbf{Malicious Action:} deliberate act by a person, organization or state that has the aim of harming the proper functioning of the healthcare process, such as reprogramming or deactivating medical devices. Such an action can be performed both from the inside and from the outside and are very relevant as they can target a large number of systems with little effort.
    \item \textbf{System Failure:} related to the malfunction of the software component of the system. Given the interconnected nature of the components, it is essential that every device, IoT and not, is always functional, since the patient's life can depend on each of them.
    \item \textbf{Network Failure:} due to the fact that everything is super-connected, even a single compromised device can generate a botnet. Critical medical devices may be very alterable and in the case of network failures the consequences are fatal.
    \item \textbf{Human Error:} they occur during network configuration, device functioning or process execution. Often the cause is linked to inadequate processes or inadequate staff training. It may seem trivial, but human error is one of the major causes of risk within a critical infrastructure.
    \item \textbf{Natural Phenomena:} usually the most devastating, but also the least likely.
\end{itemize}
In the healthcare sector, such threat sources lead, voluntarily or not, to very serious problems that could endanger the lives of patients. Session medjaking, i.e. the hijacking of a medical devices, may compromise the proper functioning of a smart pacemaker or stop an oxygen pump for a patient in resuscitation. A DoS (and its distributed counterpart DDoS) may disrupt a hospital's reservation system or communication between its departments. A Ransomware may encrypt the entire database and prevent staff from viewing medical records and personal data of patients. A data breach may expose the so-called \textit{Personally Identifiable Information (PII)}, sensitive data of patients, but also of doctors, nurses and specialized personnel working within the hospital. Healthcare data has become the target for hackers due to its demand. Their value is now higher than that of credit card information that are sold on the black market. PII leaks undermine patient privacy and lead to the proliferation of targeted advertisements, identity theft and insurance fraud. In addition, unauthorized access to personal data, with the modification or injection of false data, may impact medical decisions. Although digitization plays a fundamental role in hospitals, processing a huge amount of sensitive data, \cite{Rajamaki2018} and \cite{Khadija2021} underline the fact that the healthcare infrastructure is not up-to-date in the field of cybersecurity, but rather very dated. The case of the WannaCry ransomware is emblematic: over 150 nations in 2017, among the most advanced in the world, were hit by a worldwide attack: healthcare infrastructure was one of the main targets of this attack. Sector studies show that only a small part of the budget is used in cybersecurity, making healthcare one of the most prone and vulnerable infrastructures to attacks, if not the most, especially in the last few years and during the COVID-19 pandemic. The likelihood of a cyber attack to occur in the healthcare is greater than in other sectors of economy. \cite{Dogaru2017} point out that the attacks may come from within the organization and in that case the resulting impact may be much greater. In fact, an internal threat has access to a greater variety of resources within the hospital if the security policies are loosely implemented. The study by \cite{Kandasamy2022}, focused on cyberattacks on hospital infrastructures in various regions of Asia, reports important data on the impact due to internal threats. The source of a large part of the cyber attacks on Asian hospitals are internal, mainly unintentional and caused by poor cyber awareness. The classic employee clicking on the wrong link. And the situation is no different in other parts of the globe. Even in countries like U.S., known to be very advanced in the field of cybersecurity, there have been numerous breaches even caused by simple phishing attacks. Phishing attacks. In 2022. No comments.
\section{Proposed solutions and considerations}
To our surprise (what a surprise), then, we found out that my blood tests aren't the only things wrong with the healthcare infrastructure. This does not mean that we have to lose faith in cyber security in hospitals. The means necessary for the consolidation of a correct cyber posture are there (we have analyzed them well in the previous chapters) and it is "only" necessary to use them properly and not to underestimate all the possible risks that technology brings with it. The solutions proposed, at least theoretical, to the multitude of problems that haunt healthcare, range from the most classic to the most extravagant ones. Obviously, even if there should be no need to repeat it, the most classic solution is to follow an internationally recognized framework. Whether it is ISO/IEC, NIST or one of its declensions is not too relevant. The important thing is that the management pays attention to every detail and that it has an overview of everything that revolves around the organization. With the increase in information traffic, the number of risks associated with sharing information has also increased and it is therefore necessary to keep an eye on this aspect as well. Giving the right value to the data managed by the healthcare infrastructure is now a minimum cybersecurity requirement: transmission and encryption protocols, as well as new models for sharing information between organizations have been proposed and continuously improved \cite{Hautamaki2020}.
An increased level of security across valuable assets as well as with respect to data exchange can be achieved with continuous monitoring and periodic review of processes \cite{Coppolino2019}. Once security policies have been implemented, they cannot be left to themselves. They must be followed and monitored constantly, improved if necessary. There was also talk of specific certifications for cybersecurity in healthcare. A standardized certification scheme would allow to have a single procedure, which would streamline the implementation processes and would allow all organizations in the sector to collaborate for common improvement \cite{Hovhannisyan2021}. We should not forget, however, slightly more recent solutions that require further studies such as the use of blockchain technologies as a deterrent to fraudulent data modification.\newline
But all of this takes a back seat if all the previous super procedures are not followed correctly. In fact, what all the studies examined have in common is that we can implement dozens of controls, install firewalls, use VPNs or boast of implementing all existing frameworks, but if the procedures are not followed to the letter there will always be the human factor that hinders us. What often happens is that the presence of too many solutions can itself be the cause of poor cybersecurity awareness. Too many procedures, too complex or difficult to assimilate, discourage personnel, specialized in IT and not, to follow the protocols, even if this could lead, as very often happens, to serious flaws in the healthcare system. The advice I would like to give, given the often banal nature of the causes of the accidents, is to invest more in training and cybersecurity awareness to avoid unnecessary risks.
\section{Dummy Example}
To conclude, I will try to make the concept of cyber risk management even clearer with an example scenario within a hospital, but applicable to any organization.\vspace{0.5cm}\newline
It is not clear why or huw, but you were assigned to the direction of the Hallux Valgus Prevention Institute (HVPI) in Velletri, a point of reference for thousands of patients suffering from this and many others serious pathologies. The facility hasn't been updated since the post-war period and the only glimmer of technology, prior to your arrival, was the secretarial phone. Your job is to bring the hospital into the new millennium. You know very well that new technologies are essential for the proper functioning of a structure of this caliber, but it is also well known that "great powers come with great responsibilities". Your only notion in cybersecurity is knowing that if something can go wrong, then it will go wrong for sure. Even more dealing with cyberspace. A quick Google search brings you to the knowledge of numerous frameworks that can assist you in this arduous task. Excellent! Now which one do you choose though? We have seen that the ultimate purpose of all frameworks is more or less identical, but the approaches used may differ. One of the main discriminators that the HVPI must take into account is the cost as it is a state institution and with few available funds. A paid certification framework can therefore be excluded in favor of an open one, but equally recognized. The choice therefore falls on the National Framework for Cybersecurity and Data Protection. You download the framework PDF and start studying.\vspace{0.5cm}\newline
Having taken note of the mechanics of the framework, you begin to implement it. The first step of the choosen framework is contextualization and since the hospital is a regulated state body, the contextualization provided by the sector regulator will be used. This contextualization, probably, will have a focus on the protection of sensitive patient data and on the availability of the IT systems on which the hospital will be based. Patients and the state will be your stakeholders and your business goal will be to prevent everything from falling apart: to avoid data breaches, to always keep medical records databases and booking services available or to protect medical devices from unauthorized access or incorrect use.\vspace{0.5cm}\newline
The contextualization aligns hospital's objectives to the available Categories, but there is the need to describe the current situation to understand the direction to take.
We start from a Current Profile almost empty and a level 2 Implementation Tier or lower, the only protection is given by the building's fire protection system. The Target Profile, on the other hand, contains all the mandatory Categories that the contextualization imposes. Nobody forbids you to add other Categories to the Target Profile that you think could improve the cybersecurity posture of the hospital: after all, you are the "mega director" and you know what is good for stakeholders. You want your hospital to be a reference point also for cybersecurity and for cybersecurity to be a backbone of the hospital system. For that reason the Implementation Tier you want to reach is at least of level 3. The availability of the database and the confidentiality of the data it contains are to be considered mandatory. Ensuring that the online booking and payment service for visits is constantly active is recommended. All information and IT processes will be monitored regularly and the (secure) sharing of information will take place both inside and outside the facility. The online parking reservation can be considered optional I would say.\vspace{0.5cm}\newline
Compared the Current Profile with the Target Profile and highlighted the gaps to be filled, it is time to implement an action plan. The budget, albeit limited, will be invested mainly in the protection of primary (digital) assets (taking for granted the safety of fundamental systems such as the electricity or surveillance network). The possibility of examine the medical records of patients from a computer screen implies the need to make this use possible only for authorized doctors. For this purpose, the installation of an authentication system (username and password) is mandatory. A version of it with two-factor authentication, on the other hand, may be superfluous or inconvenient for the specialist who will use it. The choice will depend on the likelihood that a doctor has his login credentials stolen or lost. The data contained in the database, however, must be encrypted to avoid data leaks and protected by firewalls and antivirus to prevent intrusions from remote, but also locally.\newline
The doctor may want to access medical records even from outside the workplace for remote diagnosis: if this is allowed by organization policy, the connection between his personal device and the internal system of the hospital must be protected from any malicious persons that may be listening with the use of encryption or VPN.\newline
The online booking system must be able to resist numerous simultaneous accesses by users, but also to detect and block malicious accesses. Penetration testing actions can help highlight any leaks in the system's incoming connections that could result in DDoS attacks or unauthorized access. The payment system, on the other hand, will most likely be outsourced to an external supplier (the hospital can't do everything on its own). In this case, the framework allows us, thanks to the Target Profile, to select the appropriate provider based on the security criteria we aspire to. Patient identity and data relating to payment methods must remain safe from any unauthorized access.\newline
Let's move on to patient-critical systems. Respirators, EKGs and similar devices can be managed via computer. Nice yes, but you have to be careful. If connected to the network, these devices can be manipulated at will by an attacker who manages to penetrate the system, modifying stats or performing unsolicited and dangerous actions for the patient. Installing security systems such as an IDS can prevent unauthorized intrusions.\vspace{0.5cm}\newline
Once prevention tools are chosen and implemented, hospital governance must ensure that they are used in the correct way. The security policies, differentiated by sector of competence, must be communicated in each department to instill a cyber culture in the staff. The weak link in any computer system is the end user. Using the word "password" as a password makes the hacker's job almost tedious. Installing a firewall on the primary server becomes useless if local access is left exposed. In addition, software can also have bugs and security flaws which could allow a potential attacker to take control. The keystone of cyber risk management is the continuous monitoring of processes. It will be important to invest even a small part of the budget in awareness campaigns towards cybersecurity for employees inside and outside the workplace. If everything went as it should, congratulations, your hospital is safe. However, the situation can change quickly. Things don't always go smoothly and it is important to always keep your guard up. 
