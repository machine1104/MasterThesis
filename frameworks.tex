\chapter{Frameworks}
We have been waiting for it for a long time, the main event. It's time to talk about frameworks! From now on we will use the term \textit{framework} to generalize the category of standards and best practices we have talked about in past chapters. It would have been nice if only one existed: less to write for me and to read for you. Unfortunately for those who have to read all this paper and luckily for all those companies that have different needs, there are many. We will see that each framework has peculiarities that distinguish it from the others and adapt it to a particular use. In this chapter I will try to summarize the main ones.
\section{NIST Cyber Security Framework}
Following the \textit{Cybersecurity Enhancement Act of 2014 (CEA)} in the United States, the \textit{National Institute of Standards and Technology (NIST)} has been in charge of identifying and developing cybersecurity risk frameworks for voluntary use by critical infrastructure owners and operators. Being the critical infrastructures...critical for Nation's security, economy and public safety and increasingly complex and connected to each other, it was necessary to protect them from the growing numbers of cybersecurity threats. What came out was the \textit{NIST Cyber Security Framework (NIST CSF)} \cite{NIST2018}. The framework is flexible and adapts to different organizations in any sector or community and considers cybersecurity risks as part of the organization’s risk management processes. Practices described in the framework can be customized depending on different threats, vulnerabilities and risk tolerances and how to apply them is left to the implementing organization. Each organization can choose independently the activities that are important to their critical service delivery and therefore prioritize the investments. Even though it was born with the purpose of protecting organizations in the USA, the framework can be adopted globally and can serve as a model for international cooperation on strengthening cybersecurity in critical infrastructures. Being technology neutral and based on global standards, guidelines, and practices the framework provides a common taxonomy and mechanism for organizations to describe their current cybersecurity posture, their target state and the steps required to reach it while assessing the progress and communicating among internal and external stakeholders. The framework does not replace an organization’s risk management process and cybersecurity program, at most it complements it! The organization can use the framework to enhance its current processes or, in absence of a cybersecurity program, as reference to establish a new one. NIST CSF is composed of three parts, \textit{Framework Core, Framework Implementation Tiers} and \textit{Framework Profiles}, that reinforces the connection between business drivers and cybersecurity activities.
\subsection{Framework Core}
The Framework Core is a set of activities that have the purpose of achieving some specific cybersecurity results. Each activity is associated with references examples that help to achieve the desired outcomes. The Core is not a checklist! Not all activities are mandatory.
\begin{figure}[H]
  \centering
  \includesvg[inkscapelatex=false,width=\textwidth]{nist-1.svg}
  \caption{Framework Core structure}
\end{figure}
\noindent
In turn, the Framework Core is divided in \textit{Functions, Categories, Subcategories} and \textit{Informative References}.
\subsubsection{Functions}
Framework Functions group the basic activities into five sections, which correspond to the steps of the Incident Response Lifecycle. They aid an organization in expressing its management of cybersecurity risk by organizing information, enabling risk management decisions, addressing threats, and improving by learning from previous activities. These functions do not have to be performed in order like a recipe for a dessert, but concurrently and continuously to form an operational culture that addresses the dynamic cybersecurity risk.
\subsubsection{Categories}
Categories are the subdivisions of a Function into groups of cybersecurity outcomes closely tied to programmatic needs and particular activities. They represent the various results that can be obtained in the respective Functions. Example (from \textit{Identify} Function).\newline\textbf{Asset Management (ID.AM)}: the data, personnel, devices, systems, and facilities that enable the organization to achieve business purposes are identified and managed consistent with their relative importance to organizational objectives and the organization’s risk strategy.
\subsubsection{Subcategories}
Subcategories are the subdivision of the subdivision of a function. The activities described concern technical and management aspects and contribute to the achievement of the objectives of the category to which they belong.
Linking to the previous example, subcategories of ID.AM are:
\begin{itemize}
\itemsep0em
    \item \textbf{ID.AM-1}: Physical devices and systems within the organization are inventoried.
    \item \textbf{ID.AM-2}: Software platforms and applications within the organization are inventoried
    \item \textbf{ID.AM-3}: Organizational communication and data flows are mapped
    \item \textbf{ID.AM-4}: External information systems are catalogued
    \item \textbf{ID.AM-5}: Resources (e.g., hardware, devices, data, time, personnel, and software) are prioritized based on their classification, criticality, and business value
    \item \textbf{ID.AM-6}: Cybersecurity roles and responsibilities for the entire workforce and third-party stakeholders (e.g., suppliers, customers, partners) are established
\end{itemize}
As easily understood, the sum of these subcategories leads to the achievement of the goal of the ID.AM category.
\subsubsection{Informative References}
Informative References list standards, guidelines, and practices that explain how to achieve the outcomes associated with each Subcategory. How to inventory physical devices and systems (ID.AM-1)? Check this long list of standards and guides!
\begin{itemize}
\itemsep0em
    \item \textbf{CISCSC} 1
    \item \textbf{COBIT 5} BAI09.01, BAI09.02
    \item \textbf{ISA 62443-2-1:2009} 4.2.3.4
    \item \textbf{ISA 62443-3-3:2013} SR 7.8
    \item \textbf{ISO/IEC 27001:2013} A.8.1.1, A.8.1.2
    \item \textbf{NIST SP 800-53Rev. 4} CM-8, PM-5
\end{itemize}
Every subcategory has its own list of references and for the complete list of categories and subcategories, see the original document \cite{NIST2018}.
\subsection{Framework Implementation Tiers}
The Framework Implementation Tiers indicate the level of cybersecurity within an organization. The higher the level, the more rigorous and sophisticated the cyber risk management processes are. The tiers do not represent maturity levels though. The transition from one tier to the next is only encouraged in the presence of a favorable cost-benefit analysis. Let's compare the four tiers with the power levels reached by \textit{Goku} in \textit{DragonBall Z}: the transition from \textit{Kaioken (Tier 1)} to \textit{Super Saiyan (Tier 2)} is very often encouraged because it brings numerous benefits without exorbitant costs, while switching from \textit{Super Saiyan 2 (Tier 3)} to \textit{Super Saiyan 3 (Tier 4)} is not recommended unless strictly necessary due to the large expenditure of energy. There is no point in implementing Tier 4 on a tech farm: I don't think cucumbers care much about their cybersecurity. The selection of the Tier is not done by chance: it is necessary to take into account current risk management practices, the work environment, legal restrictions, the objectives and all the requirements of the organization, making sure that the chosen tier fully satisfies them, is easy to implement and that it actually works. Tier selection and designation naturally affect Framework Profiles.
\subsubsection{Tier 1: Partial}
Cybersecurity risk management is not a priority and the processes are not formalized. Threats are often handled as they happen and cyber risk management is implemented case-by-case. Cybersecurity related information may not be shared within organization and definitely not with external entities, with which it does not even collaborate. Organization is not aware of cyber risks related to services it uses or provides. Clueless.
\subsubsection{Tier 2: Risk Informed}
Risk management processes are approved by management, but may not be applied organization-wide. Cybersecurity may be considered in organizational objectives at some, but not all, level of organization and information sharing is done, but on a informal basis within the organization. The organization understands its role in the larger ecosystem with respect to either its own dependencies or dependents, but not both. It collaborates with external entities, but may not share information with them. Unlike tier 1, tier 2 organization is aware of the cyber risks associated with products and services it uses or provides, it just doesn't care that much. Experienced noob.
\subsubsection{Tier 3: Repeatable}
Cybersecurity is now serious. All practices are approved, expressed as policies, regularly updated according to the objectives of the organization and applied organization-wide. The organization continuously monitors its assets for any threats and information sharing is done at every level. The organization not only understands its role in the larger ecosystem, but can also actively contribute to the community. Collaboration with external entities also involves the sharing of internally generated information and the organization formally takes action to manage any risks related to the products and services it uses and provides. Smartass.
\subsubsection{Tier 4: Adaptive}
Cybersecurity practices, which now also include lessons learned and predictive indicators, are continuously developing to adapt to new technologies and respond promptly to new threats. Cybersecurity is taken into consideration when making decisions regarding organizational objectives, so much so that the budget is based on an understanding of the current and predicted risk environment and risk tolerance. The sharing of information, whether internal or external, now includes real-time or near real-time information. The organization plays an active role in analyzing the risks of the sector in which it operates, often communicating proactively, using formal (e.g. agreements) and informal mechanisms and methods to maintain strong supply chain relationships. Grand Master.
\subsection{Framework Profiles}
The Framework Profile (“Profile”) is the alignment of the Functions, Categories, and Subcategories with the business requirements, risk tolerance, and resources of the organization. The Profile has two functions: to describe the current state of the organization (Current Profile), which corresponds to the current results in the cybersecurity field, or the objectives to be achieved (Target Profile). The Framework has no Profile templates, leaving full freedom to organizations, which may even have different profiles for their different components. The comparison between Current Profile and Target Profile allows you to reveal the gaps to be bridged to reach the desired state and to create a dedicated roadmap. Prioritizing the mitigation of gaps is driven by the organization’s business needs and risk management processes. This risk-based approach enables an organization to gauge the resources needed (e.g. staffing, funding) to achieve cybersecurity goals in a cost-effective and prioritized manner.
\subsection{Use cases}
What do we do with this stuff? A lot of things.
\subsubsection{Review}
The simplest thing to do with the Framework is to use it to compare an organization's current cybersecurity activities with those outlined in the Framework Core. Creating a Current Profile allows the organization to align its activities with the five high-level Functions of the Framework and to understand if it is already achieving the desired outcomes or if it needs to improve to achieve them and how to do it. Knowing the current state of an organization allows to prioritize resources and develop an action plan to strengthen existing cybersecurity practices. The Current Profile and the five Functions do not replace a risk management process, but distills the fundamental concepts of cybersecurity risk and assess how identified risks are managed.
\subsubsection{Create or Improve}
The Framework can be used to create a completely new cybersecurity program or improve an existing one. The following steps should be repeated as necessary to always update cybersecurity practices.
\begin{enumerate}
    \item \textbf{Prioritize and Scope:} The first thing to do is to identify the business objectives and high-level priorities of the organization. Knowing this, strategic decisions regarding cybersecurity can be made. In an organization there may be more business lines, which may have different needs and the Framework can be adapted to support each of them. The risk tolerance of the different business lines can be reflected in a target Implementation Tier.
    \item \textbf{Orient:} Having understood the objectives and determined the scope of the cybersecurity program, the organization identifies related systems and assets, regulatory requirements, and overall risk approach. Then the organization consults sources to identify threats and vulnerabilities applicable to those systems and assets.
    \item \textbf{Create a Current Profile:} The organization indicates which Category and Subcategory outcomes from Framework Core are currently achieved and build a Current Profile on them.
    \item \textbf{Conduct a Risk Assessment:} The assessment may follow the organization's risk management processes or be based on previous risk assessment activities. By analyzing the operational environment and cyber threat information from internal and external sources, the organization must evaluate the likelihood and impact of potential cyber threats.
    \item \textbf{Create a Target Profile:} The organization puts its Target Profile in writing, also creating specific Categories and Subcategories to account for unique risks and requirements, whether internal to the organization or external such as sector entities, customers and business partners. The Target Profile should appropriately reflect criteria within the target Implementation Tier.
    \item \textbf{Determine, Analyze and Prioritize Gaps:} Current and Target Profile are compared and the gaps are highlighted. An action plan is created to address the gaps, prioritizing them and reflecting mission drivers, costs and benefits and risks, in order to reach the Target Profile. Using Profiles, the organization can make informed decisions about cybersecurity, such as determining the resources needed to address its gaps.
    \item \textbf{Implement Action Plan:} The plan is implemented to adjust organization current cybersecurity practices and achieve the Target Profile. As previously mentioned, the Framework proposes Information References dedicated to the various Categories and Subcategories, but the organization should decide which of these, or perhaps others specific to the sector, works best for it.
\end{enumerate}
The steps can be repeated non-stop, sky's the limit. An organization can repeat the entire process in a loop or focus on a single step. Repeating the creation of the Current Profile iteratively, for example, allows the organization to monitor progress towards the Target Profile.
\subsubsection{Communicate and Decide}
The Framework is also a tool that allows clear communication between the different stakeholders. Two trivial examples are the Current and Target Profile. The Current Profile may express the organization's cybersecurity posture and report the results achieved to a customer. The Target Profile may express the minimum requirements of an organization to a potential external service provider or act as a baseline for an entire sector. Communication is especially important among stakeholders up and down cyber supply chains.
\begin{figure}[H]
  \centering
  \includesvg[inkscapelatex=false,width=\textwidth]{nist-2.svg}
  \caption{Cyber Supply Chain relationships}
\end{figure}
\noindent
Supply chains are complex and interconnected, involve multiple levels of organizations and it is unthinkable that every organization uses a proprietary language: it would be chaos. A well-structured cybersecurity process throughout the supply chain allow a solid cyber supply chain risk management (SCRM): more specifically, cyber SCRM addresses both the cybersecurity effect an organization has on external parties and the cybersecurity effect external parties have on an organization. Taking into consideration the so-called "suppliers" and "buyers" and their requirements is important to avoid the emergence of threats: if you buy a security system from a company that identifies itself in Tier 1, do not complain if your data is stolen and sold on the deep web. In fact, sometimes it may not be possible to impose a set of cybersecurity requirements on the supplier and it is therefore necessary to decide which of the many best meets the organization's requirements. Once a product or service is purchased, the Profile also can be used to track and address residual cybersecurity risk.
\subsubsection{Protect}
An aspect that should not be underestimated is the protection of privacy: privacy and cybersecurity go hand in hand. Activities that could have cybersecurity as their purpose, if applied in the wrong way, can undermine privacy and civil liberties when dealing with personal information. Collection of personal information may turn into over-collection, endangering more personal data than expected. The government and its agents, in the context of critical infrastructures, have the obligation to monitor the compliance of cybersecurity activities with privacy laws, regulations and Constitutional requirements. For these reasons, privacy must play an important role in all processes of the organization and its implications must be considered among the risks and related responses. Internal and external staff must be informed and trained on the privacy policies and report to appropriate management. There must be processes that monitor and review the organization's activities, which involve personal information: detect anomalous activity, assess how and when personal information is shared outside the organization and review mitigation efforts.
\section{National Framework for Cybersecurity and Data Protection}
We now move to Italy, where in 2015 the National Framework for Cybersecurity was presented (edited by my former professor Roberto Baldoni, modestly) \cite{CISSAP2015}. The Framework is inspired by the one illustrated in the previous section, sharing Core, Tier and Profiles, but at the same time introduces new fundamental elements, new implementation mechanics for SMEs and from 2018 regulates the processing and circulation of personal data according to the GDPR \cite{CISSAP2019}. Just like the NIST CSF, however, the Framework cannot be considered a tool for complying with current regulations, but only a tool to help organizations define a path towards cybersecurity and data protection or to guide the necessary continuous monitoring activities.
\subsection{NEWS!}
The first change with respect to the NIST CSF is the introduction of 1 new Category and 9 new Subcategories (identified by the prefix “DP-“ and highlighted in yellow in the following table) concerning the protection of personal data, which were not sufficiently understood by the Subcategories already present in the original Framework.
\begin{table}[H]
\begin{tabularx}{\textwidth}{|c|c|X|} 
\hline
{\cellcolor{dummy-cyan}}\textbf{\textcolor{white}{FUNCTION}} & 
{\cellcolor{dummy-cyan}}\textbf{\textcolor{white}{CATEGORY}} &
\multicolumn{1}{|c|}{{\cellcolor{dummy-cyan}}\textbf{\textcolor{white}{SUBCATEGORY}}}
\\ 
\hline
\multirow{20}{*}{\begin{tabular}[c]{@{}c@{}}{\textbf{IDENTIFY}}\\{\textbf{ID}}\end{tabular}} &
\multirow{3}{*}{\textbf{ID.AM}} &
{\cellcolor{dummy-yellow}}{\textbf{DP-ID.AM-7:} Roles and responsibilities relating to the processing and protection of personal data are defined and disclosed for all personnel and for any relevant third parties (e.g. suppliers, customers, partners)}\\
\cline{3-3} & & {\cellcolor{dummy-yellow}}{\textbf{DP-ID.AM-8:} The processing of personal data is identified and cataloged}\\
\cline{2-3} & \textbf{ID.RA} & {\cellcolor{dummy-yellow}}{\textbf{DP-ID.RA-7:} An impact assessment on the protection of personal data is carried out}\\
\cline{2-3} & {\cellcolor{dummy-yellow}} & {\cellcolor{dummy-yellow}}{\textbf{DP-ID.DM-1:} The data life cycle is defined and documented}\\
\cline{3-3} & {\cellcolor{dummy-yellow}}  & {\cellcolor{dummy-yellow}}{\textbf{DP-ID.DM-2:} The processes concerning the information of the interested party regarding the processing of data are defined, implemented and documented}\\
\cline{3-3} & {\cellcolor{dummy-yellow}} & {\cellcolor{dummy-yellow}}{\textbf{DP-ID.DM-3:} The processes for collecting and revoking subject's consent to data processing are defined, implemented and documented}\\
\cline{3-3} & {\cellcolor{dummy-yellow}} & {\cellcolor{dummy-yellow}}{\textbf{DP-ID.DM-4:} The processes for exercising the rights (access, rectification, cancellation, etc.) of the interested party are defined, implemented and documented.}\\
\cline{3-3} & \multirow{-15}{0.2\textwidth}{{\cellcolor{dummy-yellow}}{\textbf{Data Management (DP-ID.DM):} Personal data are processed through defined processes, in accordance with the reference regulations.}} & {\cellcolor{dummy-yellow}}{\textbf{DP-ID.DM-5:} Data transfer processes in an international context are defined, implemented and documented}\\
\hline
\begin{tabular}[c]{@{}c@{}}{\textbf{RESPOND}}\\{\textbf{RS}}\end{tabular} & \textbf{RS.CO} & {\cellcolor{dummy-yellow}}{\textbf{DP-RS.CO-6:} Incidents that involve personal data breaches are documented and, if necessary, the relevant authorities and interested parties are informed}\\
\hline


\end{tabularx}
\end{table}\noindent
The second news is the introduction of the \textit{Priority Levels}.
Priority Levels make it possible to support organizations in defining an implementation program to reach a Target Profile by giving priority to the interventions that most reduce the risk levels to which they are subjected. Actually this procedure was also recommended in the NIST CSF, but no Priority Levels were explicitly assigned to the individual Subcategories. The priority is based on two factors: effectiveness and simplicity of implementation. The effectiveness, in particular, is determined on the ability to act on one or more key factors of the cyber risk: exposure to threats, probability of their occurrence and consequent impact. There are three Priority Levels:
\begin{itemize}
    \item \textbf{High:} Interventions that make it possible to significantly reduce one of the three key factors of the cyber risk, to be implemented independently of the complexity of their implementation.
    \item \textbf{Medium:} Interventions that allow to achieve a reduction of one of the three key factors of cyber risk and which are generally also simple to implement.
    \item \textbf{Low:} Interventions that allow to achieve a reduction of one of the three key factors of cyber risk, but generally with high implementation complexity.
\end{itemize}
Each organization, in the adoption of the Framework or during the Contextualization activity (we'll talk about it later), could redefine specific priority levels for each subcategory. Connected to this, the \textit{Maturity Levels} allow the organization to provide a measure of the maturity of a security process, the maturity of implementation of a specific technology or a measure of the amount of adequate resources used for the implementation of a given subcategory. Simplistically, they indicate how long it is to reach the target level of the single Subcategory. The levels must be defined in increasing progression, from the lowest to the highest. Each level must provide for practices and incremental checks with respect to the lower Maturity Level. An organization may have different Maturity Levels for different subcategories, but it must be at least equal to that of all related security practices implemented.\newline
Completely new compared to the NIST CSF are the \textit{Contextualization Prototypes}. These prototypes are real templates that can be used for contextualization in specific sectors. Each prototype divides the Subcategories into three implementation classes and, for each of them, can define a priority level for their implementation. Each Subcategory can be defined as:
\begin{itemize}
    \item \textbf{Mandatory: } It MUST be included in all contextualizations that implement the prototype.
    \item \textbf{Recommended: } Its inclusion is recommended in all contextualizations that implement the prototype.
    \item \textbf{Optional: } Its inclusion in the contextualizations that implement the prototype is left to the free choice of those who define such contextualizations.
\end{itemize}
Each prototype also has its own implementation guide, a kind of leaflet, which describes the application context, further constraints on the selection of Subcategories and definition of the Priority Levels and a list of security checks that will be appropriately organized in the different Maturity Levels.
A Contextualization Prototype represents, to all intents and purposes, a fundamental element on the basis of which it is possible to build a new contextualization or which can be adopted and implemented in an existing contextualization. It is therefore important to emphasize that Contextualization Prototypes do not replace contextualizations, but provide a support tool that facilitates the creation and updating of a contextualization, through the composition of multiple prototypes.
\subsection{Operating instructions}
The use of the Framework is achieved through two fundamental activities described below in this section: the Contextualization of the Framework to a specific application area and the application of the Framework to an organization.
\subsubsection{Contextualization}
Contextualizing the Framework for an application area (e.g. a production sector or a homogeneous category of organizations) means specifying its Core by selecting the relevant Functions, Categories and Subcategories, and defining the Priority and Maturity Levels for the selected Subcategories: in simple term adapt the Framework to the needs of the organization. Production sector, type of employees, size and location in the territory of an organization determine the Subcategories and levels of Priority and Maturity that will be implemented. The steps are simple:
\begin{enumerate}
\itemsep0em
    \item Select Function, Categories and Subcategories that are relevant to the organization based on all or some of the above elements (production sector, type of employee etc.).
    \item Define the Priority Levels for the implementation of the selected Subcategories.
    \item Define guidelines for at least high priority Subcategories.
    \item Specify Maturity Levels at least for high priority Subcategories.
\end{enumerate}
One recommendation: high priority Subcategories must always be implemented, at least at the minimum level of maturity.\newline
\begin{figure}[H]
  \centering
  \includesvg[inkscapelatex=false,width=\textwidth]{ita-1.svg}
  \caption{Framework Contextualization through the implementation of prototypes}
\end{figure}
\noindent
The task of defining the Contextualization is usually left to the organization itself, but there are also cases in which this definition is provided by a sector association, a regulator or any actor who is able to identify and apply to the Contextualization a set of characteristics belonging to one or more organizations. As long as he knows what he is doing for God's sake. It is also possible to implement more prototypes and then refine the resulting Contextualization with respect to its specificities. The process of implementing a prototype, represented in the figure, is as follows. The Subcategories indicated as mandatory in the prototype are selected without discussion, while the recommended ones are evaluated case by case in consideration of the specific characteristics of the application area. Any constraints documented in the application guide must be applied. Always taking these constraints into account, for each Subcategory, it is necessary to select a Priority Level, preferably at least equal to that indicated in the prototype. The security checks indicated in the implementation guide can be integrated into the guidelines for the application of the Contextualization. Repeated this process for all the prototypes of interest, the resulting contextualization can be further specialized, if the need arises.
\subsubsection{Application}
The main objective of the Framework is to provide interested organizations with a tool to support the process of managing and treating cyber risk. We have already heard this story. The steps for implementation are in fact almost identical to the 7 steps required by NIST CSF for the creation of a cybersecurity program or for the improvement of one already present. However, two further steps are added, respectively to the first and last position. The new first step is obviously the identification of the Contextualization. If the organization belongs to a regulated sector, it will adopt the Contextualization provided by the regulator or choose from the available prototypes. Otherwise, it will choose the one it likes most from the available Contextualizations. The last step, on the other hand, concerns performance monitoring. It specifies that it is necessary to define monitoring metrics capable of highlighting operating costs and assessing the performance of Current and Target Profiles on a periodic basis.\newline
Finally, it is envisaged that the Framework can be used to assess the Maturity Level of cybersecurity activities and processes. To do this, the steps are the same except for the step relating to the risk assessment as it has already been performed.
\subsection{Essential Cybersecurity Checks}
Believing that only big tech companies are the target of cyber attacks is a mere illusion, a joke, a gag, a story you tell yourself to believe that the world is a better place. This often leaves medium, small and micro enterprises totally unprepared to face cyber threats and increases the cyber risk within the entire production chain in which they operate. It is also true, however, that the director of a small organization will probably spit in your eye if you propose a cybersecurity program that is too complex or expensive. For this reason, 15 Essential Cybersecurity Checks have been proposed \cite{CISSAP2016}, derived from the Framework, but simpler to implement. They obviously do not ensure an adequate level of security, but they represent a basis from which a progressive improvement path must start up to align with the Framework. With regard to data protection and the new Subcategories implemented, the Essential Checks can be grouped as explained in the following paragraphs.
\subsubsection{Device and software Inventory}
This thematic contains the first four Essential Checks. Among these, the last two are related to two of the new Subcategories introduced by the Framework.
The Subcategory DP-ID.AM-7 concerns the definition of roles and responsibilities related to data protection and is therefore related to the Essential Check 4 which provides for the appointment of a contact person who is responsible not only for the protection of IT systems, but also information. Correct protection of information cannot ignore the protection of personal data against unlawful processing and violations. Subcategory DP-ID.AM-8 concerns the identification and cataloging of personal data processing and is therefore related to the Essential Check 3 which requires the identification of information, data and critical systems for the organization. It is clear that personal data must be considered critical data for the organization and must be identified and cataloged on a par with other types of data and critical systems.
\begin{table}[H]
\begin{tabularx}{\textwidth}{|>{\centering\arraybackslash}p{0.25\textwidth}|c|X|} 
\hline
{\cellcolor{dummy-cyan}}\textbf{\textcolor{white}{THEMATIC}} &
\multicolumn{2}{|c|}{{\cellcolor{dummy-cyan}}\textbf{\textcolor{white}{ESSENTIAL CYBERSECURITY CHECKS}}}\\ 
\hline
\multirow{10}{*}{\begin{tabular}[c]{@{}c@{}}{Device and software}\\{Inventory}\end{tabular}} & 1 & {An inventory of systems, devices, software, services and IT applications in use within the company perimeter exists and is kept up-to-date.}\\ 
\cline{2-3} & 2 & {The web services (social networks, cloud computing, e-mail, web space, etc.) offered by third parties to which you have registered are those strictly necessary.}\\ 
\cline{2-3} & 3 & {Critical information, data and systems for the company are identified so that they are adequately protected.}\\ 
\cline{2-3} & 4 & {A contact person has been appointed who is responsible for coordinating the management and protection of information and IT systems.}\\ 
\hline
\end{tabularx}
\end{table}
\subsubsection{Governance}
The thematic "Governance" collects only one Essential Check, which requires identification and compliance with the applicable legislation relating to cybersecurity. Even if there is no explicit mention of data protection, it goes without saying that this Check can and must be extended to comply with the rules relating to data protection. This Essential Check is related to a pre-existing Subcategory (ID.GV-3 to be precise) which is already formulated in the NIST CSF in a sufficiently general way to include both elements of cybersecurity and elements of data protection.
\begin{table}[H]
\begin{tabularx}{\textwidth}{|>{\centering\arraybackslash}p{0.25\textwidth}|c|X|} 
\hline
{\cellcolor{dummy-cyan}}\textbf{\textcolor{white}{THEMATIC}} &
\multicolumn{2}{|c|}{{\cellcolor{dummy-cyan}}\textbf{\textcolor{white}{ESSENTIAL CYBERSECURITY CHECKS}}}\\ 
\hline
Governance & 5 & {The laws and/or regulations with relevance in terms of cybersecurity that are applicable for the company are identified and respected.}\\ 
\hline
\end{tabularx}
\end{table}
\subsubsection{Identity Management, Protection, Training and Awareness}
The Essential Checks from 6 to 13 mainly impact Subcategories of the Function Protect and, while not strongly related to any of the new Subcategories, concern the adoption of effective security measures both against cyber risk and against that related to data protection. No protection for cyber systems means no protection for the data they manage.
\begin{table}[H]
\begin{tabularx}{\textwidth}{|>{\centering\arraybackslash}p{0.25\textwidth}|c|X|} 
\hline
{\cellcolor{dummy-cyan}}\textbf{\textcolor{white}{THEMATIC}} &
\multicolumn{2}{|c|}{{\cellcolor{dummy-cyan}}\textbf{\textcolor{white}{ESSENTIAL CYBERSECURITY CHECKS}}}\\ 
\hline
\begin{tabular}[c]{@{}c@{}}Malware\\Protection\end{tabular} & 6 & {All devices that allow it are equipped with regularly updated protection software (antivirus, antimalware, etc.).}\\ 
\hline
\multirow{7}{*}{\begin{tabular}[c]{@{}c@{}}{Password and}\\{account Management}\end{tabular}} & 7 & {The passwords are different for each account, of adequate complexity and the use of the most secure authentication systems offered by the service provider is evaluated (eg. two-factor authentication).}\\ 
\cline{2-3} & 8 & {Personnel authorized to access, remotely or locally, the IT services have personal accounts that are not shared with others; access is suitably protected; old accounts that are no longer used are deactivated.}\\ 
\cline{2-3} & 9 & {Each user can only access the information and systems that they need and/or are responsible for.}\\ 
\hline
{Training and Awareness} & 10 & {The personnel is adequately sensitized and trained on the risks of cybersecurity and on the practices to be adopted for the safe use of company tools (eg. recognize e-mail attachments, use only authorized software, etc.). The organization's top management takes care to provide all company personnel with the necessary training to provide at least the basic notions of safety.}\\ 
\hline
\multirow{5}{*}{{Data Protection}} & 11 & {The initial configuration of all systems and devices is carried out by expert personnel, responsible for their safe configuration. The default login credentials are always replaced.}\\ 
\cline{2-3} & 12 & {Backups of critical information and data for the organization (identified in control 3) are periodically performed. Backups are stored securely and periodically checked.}\\ 
\hline
{Network Protection} & 13 & {The networks and systems are protected from unauthorized access through specific tools (eg. firewalls and other anti-intrusion devices/software).} \\ 
\hline
\end{tabularx}
\end{table}
\subsubsection{Prevention and Mitigation}
The last two Essential Checks are part of the thematic "Prevention and Mitigation". Always focusing on data protection, the Essential Check 14 relates to the new Subcategory DP-RS.CO-6. This check concerns the actions to be taken in the event of cyber incidents such as contacting the security officers. Among those responsible for coordinating incident response actions, there should also be a data protection officer. This person should be promptly informed to ascertain whether or not there has been a violation of personal data and, if necessary, take response actions, including those provided by the applicable legislation, such as, for example, inform the reference authorities and possibly communicate the violation to interested parties (e.g Google informing you with "Someone knows your password" email).
\begin{table}[H]
\begin{tabularx}{\textwidth}{|>{\centering\arraybackslash}p{0.25\textwidth}|c|X|} 
\hline
{\cellcolor{dummy-cyan}}\textbf{\textcolor{white}{THEMATIC}} &
\multicolumn{2}{|c|}{{\cellcolor{dummy-cyan}}\textbf{\textcolor{white}{ESSENTIAL CYBERSECURITY CHECKS}}}\\ \hline
\multirow{5}{*}{\begin{tabular}[c]{@{}c@{}}{Prevention and}\\{Mitigation}\end{tabular}} & 14 & {In the event of an incident (eg. an attack or malware is detected) the security officers are informed and the systems are secured by expert personnel.}\\ 
\cline{2-3} & 15 & {All software in use (including firmware) are updated to the latest version recommended by the manufacturer. Obsolete and no longer updatable devices or software are discontinued.}\\
\hline
\end{tabularx}
\end{table}
\subsection{GDPR Contextualization}
It is now clear that this Framework is very concerned with data protection and the GDPR in general and this section describes what is needed to contextualize the Framework and grasp the fundamental elements of the Regulation.
First of all it is necessary to know and understand the main changes introduced by the Regulation. \textit{Accountability} is one of the cardinal principles of the GDPR and concerns all those subjects (controller and processor) who deal with personal data. The data controller must, on the one hand, implement the provisions of the Regulation and, on the other hand, be able to prove the compliance of the treatments with the same Regulation. It is also necessary to adopt a risk-based approach, that is to consider the risks that could arise for the rights and freedoms of the data subjects. For this purpose the following elements have been proposed:
\begin{itemize}
    \item \textbf{Data protection \textit{by design} and \textit{by default:}} These principles, of an innovative nature, aim to determine a data protection process that takes into consideration the entire life cycle of the treatment and in particular its prodromal phases. Data protection by design implies that the technical and organizational measures adopted by the data controller are consistent with the regulatory constraints from the outset. Data protection by default, on the other hand, concerns the fact that, by default, only strictly necessary personal data are processed.
    \item \textbf{Data protection Impact Assessment (DPIA):} Before proceeding with the processing, the controller can and, under certain conditions, must carry out an impact assessment on the protection of personal data. In summary, it is an analysis of the risks for the rights and freedoms of the data subjects on the basis of which it is possible to determine probabilities, threats, impacts and security measures capable of mitigating the risk detected.
    \item \textbf{Processing activities register:} The Register of processing activities is an important tool that allows the controller to have an always updated picture of the processing within the organization for the purposes of the proper management of personal data. For each processing activity, the types of data, the technical and organizational measures adopted, the categories of data subjects, the purposes of the processing and the legal bases must be reported in this register.
    \item \textbf{Treatment security:} The GDPR requires the adoption of adequate security measures, technical and organizational, for the protection of personal data, taking into account the costs of implementation and the risks to the rights and freedoms of the data subjects. Among the elements to consider are encryption and pseudonymisation, the ability to permanently ensure the confidentiality, integrity, availability and resilience of processing systems and services, the ability to restore the availability and access of personal data. in the event of an accident and a procedure for testing and verifying the effectiveness of technical and organizational measures.
    \item \textbf{Data breach notification:} The violation of personal data entails, if it presents a risk to the rights and freedoms of the interested party, the obligation on the part of the holder to notify the Authority within the pre-established deadlines. In some particularly serious cases, communication must also be made to the interested party in such a way as to allow him to protect himself.
\end{itemize}
The Regulation also provides for different roles in the field of data protection, each with its own responsibilities.
\begin{itemize}
    \item \textbf{Data Controller:} Natural or legal person (including public bodies and authorities) who determines the purposes and means of the processing of personal data. The controller must ensure the protection of personal data by adopting appropriate technical and organizational security measures as well as imparting appropriate instructions to anyone acting under his responsibility. The controller, in fact, can make use of the help of other internal and external subjects. According to the accountability principle, it must also be able to demonstrate that the processing is carried out in accordance with the provisions of the Regulation.
    \item \textbf{Data Processor:} Natural or legal person, public authority, service or other body that processes personal data "on behalf" of the data controller. The processor must present sufficient guarantees with respect to the technical and organizational measures implemented and, with respect to the activities within its competence, carry out processing activities only for the purposes indicated by the controller.
    \item \textbf{Data Protection Officer (DPO):} Although the figure of the DPO does not represent a novelty in the European scenario, it constitutes an important innovative element in the national regulatory landscape. It must be designated, if the conditions are met, on the basis of the skills (technical, IT, legal, etc.) regarding the protection of personal data and its tasks can be summarized as: assistance to those who carry out the treatment, surveillance of compliance with the legislation on data protection and collaboration for the guarantors and interested parties.
\end{itemize}
The GDPR provides for an active participation of the data subjects in the treatment process of their personal data. Their rights, in fact, can be considered as articulations of the more general right of "information self-determination", intended as the "power to govern the flow of own information". In order for the data subject to actively participate, it is necessary that he be informed in a concise, accessible and understandable manner, with the use of simple and clear language. Furthermore, he must be informed about the identity of the data controller, data processor and DPO and all the information necessary for correct and transparent data processing. For a complete contextualization prototype of the GDPR, with colored tables and nice examples, I refer you to the original document (§ 3.2 \cite{CISSAP2019}).
\section{ISO/IEC 27005}
ISO and \textit{IEC (International Electrotechnical Commission)} together form a specialized consortium for worldwide standardization. Even national bodies are part of it and all together have the goal of developing international standards. In our case we are dealing in particular with the 27005 standard \cite{ISO/IEC2018} which provides guidelines for information security risk management. Compared to the two previous frameworks, the look changes, but not the content. There is only a small difference! NIST CSF and its Italian counterpart are totally free and do not require certification of any kind. Many ISO/IEC standards including this one, however, are paid and so are their certifications. Even the document I'm getting the information from \textit{would} be. However, it must be said that this framework is applicable to any type of organization and maybe the money it costs is worth it. The main activities and the process flow of the Framework are illustrated in figure 4.4. As you can see it is practically identical to the one illustrated a few chapters ago when we talked about risk management in general on the basis of ISO 31000. The figure shows how the process can be iterative for both risk assessment and risk treatment. The iterativity of the risk assessment can increase depth and detail of the assessment and provide a good balance between minimizing the time and effort spent in identifying controls, while still ensuring that high risks are appropriately assessed. In the same way it may happen that the risk treatment is not immediately satisfactory: in this case the risk treatment activities can be repeated or even the risk assessment can be restarted with new context parameters.
\begin{figure}[H]
  \centering
  \includesvg[inkscapelatex=false,width=\textwidth]{isoiec-1.svg}
  \caption{ISO/IEC 27005 information security risk management process}
\end{figure}
\noindent
\subsection{Context Establishment}
Internal and external context regarding information security risk management are established taking in input all relevant information about organization, setting basic criteria for cyber risk management, defining scope and boundaries and preparing the organization to support all related activities.
\subsubsection{Basic Criteria}
Different risk management approaches can be applied, also different for each iteration. The approach depends on the scope and objectives of risk management and should be selected or developed from scratch that addresses basic criteria such as: \textit{risk evaluation criteria, impact criteria} and \textit{risk acceptance criteria}.
\begin{itemize}
\itemsep 0em
    \item \textbf{Risk evaluation criteria:} developed for evaluating the risk. What else did you expect? These criteria must consider the value of the business process and all assets involved as much as their criticality. They must evaluate also availability, confidentiality and integrity along with stakeholders' expectations, perceptions and consequences.
    \item \textbf{Impact criteria:} evaluate how much an information security-related event costs in terms of degree of damange and/or money. Impact depends on the importance of target asset, the type of breach (loss of confidentiality, integrity and availability) and whether the event stops business processes or disrupts deadlines causing a damage of reputation.
    \item \textbf{Risk acceptance criteria:} define the level of accepted risk within organization. Often these depends on policies, objectives and interests both of organization and stakeholders. The criteria may include multiple thresholds and express a desired target level of risk leaving the decision of accepting higher level of risk to senior managers. Different risks may have different criteria and they can be expressed as the ratio of profit (any kind of profit) to the estimated risk. They can also include requirements for future improvements i.e. accept an higher risk ensuring that it will be reduced within a defined period.
\end{itemize}
Additionally, the organization should assess whether necessary resources are available to perform the assessment, establish a risk treatment plan, define and implement policies and procedures and monitor controls and process.
\subsubsection{Scope and boundaries}
I want to use a metaphor to explain scope and boundaries. Suppose that the organization we have to protect is your flat. The scope is this case is equal to every square meters in which you live. Knowing the scope ensure that all relevant assets (e.g. your precious collection of Pokemon cards) are taken into account during risk assessment. Boundaries, instead, are represented by the perimeter of your flat, in particular by every connection with the outside (windows and doors) that may be source of risk (e.g. a thief). When defining scope and boundaries an organization should determine and consider the environment it operates in and its relevance to the risk management process, its functions and structure, information assets and their locations, interfaces and policies. Whatever it is needed to describe the field in which the organization operates.
\subsubsection{Organization for information security risk management}
A solid structure for managing information and responsibilities is fundamental in the risk management process. The organization is responsible for developing a management process that suits itself and its stakeholders, appropriately identified and analyzed. For this purpose it is necessary to define roles and related responsibilities for all internal and external parties and to establish all possible relationships that may exist between them. The organization also has the task of identifying the information that needs to be kept, but also defining decision-making paths for the future. All of this must be approved by the appropriate managers within the organization.
\subsection{Risk Identification}
The risk assessment aims to evaluate the assets (in our case information assets), identify possible threats and vulnerabilities, ascertain about existing controls and their effectiveness on possible risks, assess the consequences (impact) and finally prioritize their resolution based on the criteria developed during context establishment. The first step of the risk assessment process is risk identification. Anything of possible interest now needs to be identified: assets, threats, controls, vulnerabilities and consequences. The aim is to understand how, where, when and why something bad, which can cause potential losses, could happen.
\subsubsection{Identification of assets}
An asset is anything that has some kind of value and therefore needs protection. Speaking of information systems, we must keep in mind that we are not just talking about hardware and software. The level of detail we will use will influence the entire risk assessment and therefore it is necessary to find the right balance, perhaps with more iterations. Each asset must be managed by someone, the asset owner, who is not the owner in the strict sense of the term, but the one who is responsible for its production, development, maintenance, use and security. The assets are divided into primary assets and supporting assets. The primary assets are usually core processes and information of activities carried out within the scope such as business processes on which the organization's mission is based or strategic information for the achievement of objectives. The supporting assets, on the other hand, are mainly hardware, software, networks, personnel, sites of the organization and the organization itself intended as a personnel structure i.e. everything that is potentially exploitable by threats that undermine primary assets.
\subsubsection{Identification of threats}
A threat can be accidental or deliberate and have human or natural origins. In any case, it is necessary to understand where they come from. The threats must be cataloged according to the different types (e.g. physical damage or compromise of information) and when necessary, it is necessary to go into the detail of individual threats. The information necessary for the identification of threats and the estimation of the likelihood of their occurrence can be obtained directly from the owners or users of the assets, from the various staff and specialists of the various sectors of the organization (e.g. facility management, legal department etc.) or by external entities such as insurance companies or government authorities. Past internal experiences and previous threat assessments must be considered in the current assessment, flanked by other threat catalogs maybe specific to the organization. When consulting a catalog, however, it is necessary to remember that threats are constantly changing and something could be missed.

\begin{table}[H]
\centering
\begin{tabularx}{\textwidth}{|c|X|c|} 
\hline
{\cellcolor{dummy-cyan}}\textbf{\textcolor{white}{TYPE}} & 
\multicolumn{1}{|c|}{{\cellcolor{dummy-cyan}}\textbf{\textcolor{white}{THREAT}}} &
{\cellcolor{dummy-cyan}}\textbf{\textcolor{white}{ORIGIN}}
\\ 
\hline
\multirow{6}{*}{Physical damage} & Fire & A, D, E\\
\cline{2-3} & Water damage & A, D, E\\
\cline{2-3} & Pollution & A, D, E\\
\cline{2-3} & Major accident & A, D, E\\
\cline{2-3} & Destruction of equipment or media & A, D, E\\
\cline{2-3} & Dust, corrosion, freezing & A, D, E\\
\hline
\multirow{11}{*}{Compromise of information} & Interception of compromising interference signals & D\\
\cline{2-3} & Remote spying & D\\
\cline{2-3} & Eavesdropping & D\\
\cline{2-3} & Theft of media or documents & D\\
\cline{2-3} & Theft of equipment & D\\
\cline{2-3} & Retrieval of recycled or discarded media & D\\
\cline{2-3} & Disclosure  & A, D\\
\cline{2-3} & Data from untrustworthy sources & A, D\\
\cline{2-3} & Tampering with hardware & D\\
\cline{2-3} & Tampering with software  & A, D\\
\cline{2-3} & Position detection  & D\\
\hline
\end{tabularx}
\caption{Threat types examples. A=Accidental, D=Deliberate, E=Environmental}
\end{table}\noindent
\subsubsection{Identification of existing controls}
To avoid wasting time, work and costs on controls that may have already been carried out, it is necessary to identify which controls are already in place throughout the process. During identification, however, it is also necessary to check them and verify that they are still operational. A control that does not work correctly can cause vulnerabilities and require complementary controls. If a control is considered insufficient or not justified, it must be determined whether it should be removed, replaced or left as it is, perhaps because there is a lack of funds to deal with it. The best way to do this is to review all documents that contain information about controls such as risk treatment implementation plans or audit results. Furthermore, an "analog" approach is also recommended: speak directly with the information security managers and users for whom these controls were created and conduct a on-site review to compare the controls implemented with the list of those that should be.
\subsubsection{Identification of vulnerabilities}
We know the assets, we know the threats, we know the existing controls. Let's put it all together and understand what can go wrong and how. The mere presence of a vulnerability does not involve a risk in itself as a threat is needed to exploit it, but it must still be monitored. Similarly, a threat that has no vulnerability may not be a risk. Controls not implemented correctly or malfunctioning can also be considered vulnerabilities. There are many methods for discovering potential vulnerabilities. They range from "simple" automated vulnerability scanning tools and penetration testing to the development, execution and evaluation of test plans to verify the effectiveness of security controls, up to the most thorough (but also most expensive) way of vulnerability assessment, the code review (check programs code, line by line, for vulnerabilities).
\subsubsection{Identification of consequences}
We are almost there. What happens if we apply the threat Z to asset X with vulnerability Y? What comes out are the consequences (or impact). A consequence can be material such as loss of production efficiency or revenue or immaterial such as loss of reputation, they can be temporary or permanent as in the case of asset destruction and they can affect one or more assets or part of them. Impact criteria must be taken into consideration when evaluating possible incident scenarios.
\subsection{Risk Analysis}
Risk analysis can be applied according to two methodologies, \textit{qualitative} or \textit{quantitative}, depending on the criticality of the assets, the extent of known vulnerabilities and the incidents that the organization has already had to face.
A qualitative risk analysis uses a very intuitive scale of qualifying attributes (e.g. low, medium, high) to describe how much attention to give to each risk based on its likelihood and the magnitude of its consequences. The advantage is that anyone can understand these scales (a high priority risk is more important than a low one, clear?), but the disadvantage is that the ratings are very subjective. Qualitative analyzes are often performed in the absence of numerical data or when resources are inadequate for a quantitative analysis and usually as an initial screening to identify risks that will be further investigated. A quantitative analysis, on the other hand, uses numerical values for both consequences and likelihood. The more the data will be accurate and complete, the higher the quality of the analysis will be. Most of the data is historical incident data whose advantage is to be very specific to the security objectives. The downside is that there is no historical data for new risks or weaknesses. A fact that should not be underestimated is that the quantitative analysis is much more complex and much more expensive than the qualitative one. Fortunately, the two can be combined.
\subsubsection{Assessment of consequences}
The impact that incidents can have on business as we have said can be expressed in qualitative or quantitative form, but any method that assigns a monetary value to the consequences is to be preferred as it generally provides more information for the decision making process. There is money at stake! The assessment takes into account two measures: the cost of replacing or repairing the information asset (if possible) and the business consequences of compromising the asset itself (including legal or regulatory ones). Asset valuation is a key factor because an incident can affect more than one asset or only a part of it and different threats and vulnerabilities have different impacts on them. During the course of the evaluation, qualitative or quantitative, an asset can have several values assigned to it, even considerably different from each other. The important thing is that at the end of the analysis each asset has a single final value that determines its cost. The impact of the accident, on the other hand, is what happens following the compromise of the asset and can have an \textit{immediate} or \textit{future} effect (financial and market consequences). An immediate impact in turn is either \textit{direct} or \textit{indirect}. An impact is direct when strictly linked to the asset such as the cost of acquiring, replacing and configuring the new asset or in the case of an information security breach. An impact is indirect if it damages the organization in other ways unrelated to the assets such as the cost of interrupted operations, potential misuse of information obtained through a security breach or the fact that the money spent to fix an asset could have been spent in better ways.
\subsubsection{Assessment of incident likelihood}
For each incident scenario and possible impact, it is now necessary to assess likelihood. Because it's okay to know that if the computer catches fire we will lose all the data on it, but how likely is that to happen? For this analysis it is necessary to take into account how often the threat in question occurs and how easy it is to exploit the vulnerability connected to it. The data used are mostly statistical and depend on factors such as the origin of the threat (deliberate or accidental), the existing controls and their effectiveness or the vulnerabilities examined individually or in aggregation. Once the likelihood value has been decided, it is compared with the impact value to give an overall final one for the risk in question.
\subsection{Risk Evaluation}
At this point the level of risks should be compared against risk evaluation criteria and risk acceptance criteria. These criteria were developed during the context establishment and it would be appropriate to give a further check to that process. Decisions are primarily based on the acceptable level of risk, but also consequences, likelihood and the degree of confidence in the risk identification and analysis should be considered as well. Many low-level risks can result in a single high-level risk. If a decision criterion (e.g. loss of confidentiality) is not important for the organization then all the risks impacting it are not relevant. If a process is considered to be of low importance then all risks related to it will have a lower consideration than risks related to more important processes. The final evaluation of a risk must provide the priority with which that risk should be treated and determine whether a risky activity should be undertaken or not.
\subsection{Risk Treatment}
Let's move on to risk treatment. Because it is not enough to write risks down on a sheet of paper to solve them. There are 4 options: \textit{risk modification, risk retention, risk avoidance} and \textit{risk sharing}. The choice on which to apply is always attributable to the risk assessment, the cost of implementation and the expected benefits. If you can greatly reduce a risk with a minimum cost, don't think twice before applying the treatment. Another story if the cost is high or the risk reduction is not so marked: in this case the managers have the task of deciding, trying to make the consequences as low as reasonably practicable and irrespective of any absolute criteria. The four options are not mutually exclusive. In many cases, in fact, the organization can decide to reduce likelihood and consequences of a risk and perhaps share or retain any residual risks. A risk treatment plan must define the order of priority with which the risks must be treated based on techniques such as risk ranking and cost-benefit analysis. It is up to managers to balance the costs of implementing controls against the budget available and this also includes deciding whether to eliminate redundant controls whose removal could not only reduce overall security, but also be more expensive than leaving them in place. Once the risk treatment plan has been defined, the residual risks are determined with an update or re-iteration of the risk assessment taking into account the new plan. If the residual risks do not meet the risk acceptance criteria, a new iteration of the risk treatment may be necessary before moving on to acceptance.
\begin{figure}[H]
  \centering
  \includesvg[inkscapelatex=false,width=\textwidth]{isoiec-2.svg}
  \caption{Risk treatment activity}
\end{figure}
\noindent
\subsubsection{Risk Modification}
The management of risk levels is subject to controls. The job of managers is to implement, remove or modify controls so that the residual risk is acceptable. The selection of controls, in order to meet the requirements identified by the risk assessment and risk treatment, must take into account the costs and timeframes for their implementation. In general, controls can provide one or more of the following types of protection: correction, elimination, prevention, impact minimization, deterrence, detection, recovery, monitoring and awareness. It is important that all costs of the control in question are balanced against the actual value of the protected asset and the return on investment in terms of risk reduction and potential to exploit new business opportunities. Furthermore there are many constraints that can affect the selection of controls. Requiring a complex password can lead an employee, not adequately trained, to write it on a piece of paper (with all its risks) or a control that seems to defeat any criticism can affect the performance of the entire process. Managers should try to identify a solution that satisfies performance requirements while guaranteeing sufficient information security. The result of this step is a list of possible controls, with their cost, benefit, and priority of implementation.
\subsubsection{Risk Retention}
What to do if the risk meets the risk acceptance criteria? Absolutely nothing.
\subsubsection{Risk Avoidance}
What do you do when you are offered to climb Mount Everest with a high probability of death and with exaggeratedly expensive equipment? You answer "no thanks".
Similarly, when the risk is too high or the costs of implementing adequate controls exceed the benefits, then you simply get around the problem by not taking the actions that involve that risk.
\subsubsection{Risk Sharing}
Sometimes, however, you have to admit your limitations and ask someone else for help. Risks can be shared with external parties such as an insurance companies or partner controllers. It should be noted that the risk can be shared, but the responsibility for the impact will always be attributed to the company.
\subsection{Risk Acceptance}
At this point, after the risk treatment plan has been established, the managers have the task of making decisions about the acceptance of any (residual) risks and formally recording them. Acceptance criteria can be much more complex than verifying whether a residual risk falls above or below a single threshold. Acceptance criteria must take into account prevailing circumstances such as the fact that the cost of the change may be too high. These circumstances indicate that the criteria are not adequate and should be revisited, but it is also true that it is not always possible to do so in a timely manner. In these cases, decision-makers can still accept the risk that does not meet the criteria, justifying the choice explicitly and taking responsibility for it.
\subsection{Risk Communication and Consultation}
During all risk management activities, information regarding them must be exchanged and shared between decision-makers and stakeholders. This information includes the existence, nature, form, likelihood, severity, treatment, and acceptability of risks. Communication is essential to understand what is at the basis of every decision made and to understand why a particular activity is required. Stakeholders' perception of risks can vary according to their assumptions, concepts and needs and their judgments on the acceptability of risks are based on it. For this reason it is important that the stakeholders' perception of risks, as well as their benefits, is identified and documented and the underlying reasons clearly understood and addressed. The organization should develop a risk communication plan for each operation, not just emergency situations, and communication activities should be performed continuously. The coordination between decision-makers and stakeholders can take the form of a committee where risks and their treatment or acceptance are discussed. Public relations and communications units are important for coordinating all risk communication tasks, especially during crisis events.
\subsection{Risk Monitoring and Review}
Keep an eye on the risks. "They like to change". The risks are not static. Threats, vulnerabilities, likelihood and consequences can change when you least expect it and it is therefore important to always stay alert and monitor the situation. An organization should ensure that new assets (or changes to them) are included in scope, the list of threats and vulnerabilities is updated periodically, and that up-to-date information is used throughout the risk management process. If the likelihood of a risk increases, then its level will also increase. The organization should review all risks regularly, and when major changes occur. The same goes for the process itself. The information security risk management process and related activities should remain appropriate to the current circumstances and any necessary improvement or action must be notified so that no risk is overlooked or underestimated and the right decisions are made. In addition, it is necessary to verify that the criteria used in the risk measurements are still valid and consistent with the business objectives, strategies and policies. The business context can also change over time and these changes need to be taken into consideration.
\section{Secure Controls Framework}
The Secure Controls Frameworks (SCF) \cite{SCF2021} is positioned halfway between a risk analysis methodology and a framework: an organization that has to comply with more standards or security requirements can find in the SCF a very useful tool. The SCF is a catalog of controls that allow the organization to design, build and maintain secure processes, systems and applications, addressing both cybersecurity and privacy. Think of the SCF as a toolkit for you to build out your overall security program domain-by-domain so that cybersecurity and privacy principles are designed, implemented and managed by default.
\subsection{Secure by Design}
An organization that follows a single standard has a very easy life: follow only the principles of that single standard and hope that they are enough to defend you. An organization with more complex security requirements, on the other hand, needs to understand which controls align with its own idea of security (or the multiple standards it intends to follow). Compliant does not mean secure, but if you design, build and maintain secure systems, applications and processes, then compliance will be a natural byproduct of those secure practices. The listing of over 1000 cybersecurity and privacy controls (a huge Excel file) is categorized into 32 domains, creating principles that an organization can use: focusing on them will grant the organization to comply with its compliance obligations. This list of principles (S|P Principles) is to be used as follows:
\begin{enumerate}
    \item Familiarize yourself with the 32 domains. It is unthinkable to check each control one by one. The domains help to understand the final objectives of the respective controls and it is not certain that these controls align with our idea of security or with our business objectives.
    \item Within the domains we want to "belong" to, we select the controls that are applicable to our organization. "Will" not always "is Power".
    \item Implement and monitor SCF controls to ensure the S|P principles are being met by your day-to-day practices.
\end{enumerate}
\subsection{Tailoring the SCF}
Suppose you are at a buffet. You have a thousand dishes at your disposal. It is unrealistic to think that you can eat every single dish. The same concept applies to SCF controls. Once you know what is applicable to you, you can generate a customized control set that gives you just the controls you need to address your statutory, regulatory and contractual obligations. Keep in mind that the SCF is a massive tool, but its effectiveness depends on how you use it. Any reference is purely coincidental. If you don't have your scope in mind you will not address your applicable compliance requirements since you are missing what is expected. Scoping is done properly only speaking with your legal, IT, project management, cybersecurity and procurement teams. This collaboration will give you a complete picture of all the applicable laws, regulations and requirements that your organization is obligated (also legally) to comply with. The SCF is fundamentally an Excel spreadsheet. What you need is only a minimal skill in filtering the requirements. Even someone like you can do it. Applicable controls are divided into 3 categories according to "must have" and "nice to have" requirements:
\begin{itemize}
    \itemsep0em
    \item \textbf{Minimum Compliance Criteria (MCC)} are the essential requirements that must be addressed to comply with laws, regulations and contracts. Don't even remotely think about ignoring them. Do not dare!
    \item \textbf{Discretionary Security Requirements (DSR)} are the optional, non-mandatory requirements that the organization can choose to comply with. Each organization can choose at its own indiscretion.
    \item \textbf{Minimum Security Requirements (MSR)} they are the equivalent of MCCs for cybersecurity and privacy.
\end{itemize}
\subsection{Target Maturity Model}
The SCF has maturity levels, called \textit{Security \& Privacy Capability Maturity Model (SP-CMM)}, which can be compared to the NIST CSF Implementation Tiers. For most organizations, the “sweet spot” for maturity targets is between CMM 2 and 4 levels, but this also depends on available resources or other business constraints. It goes beyond just the cybersecurity and privacy teams dictating targets.
\begin{figure}[H]
  \centering
  \includesvg[inkscapelatex=false,width=\textwidth]{scf-1.svg}
  \caption{Maturity Levels}
\end{figure}
\noindent
\subsubsection{CMM 0 – NOT PERFORMED}
This level of maturity is referred to as "non-existence practice" and doesn't sound very good. In practice, although there is a due to law, regulation or contractual obligation, the controls are not carried out. To define this negligent is an understatement. Be thankful they don't beat you.
\ subsubsection {CMM 1 - PERFORMED INFORMALLY}
At this level the controls lack completeness and consistency. Security practices are carried out as it happens and as needed. Performance depends on individual knowledge and effort. CMM 1 is also considered negligent, because things are done well or not done at all. What happens in practice is that the IT support role only focuses on “break/fix” work or there is no management focus.
\subsubsection{CMM 2 – PLANNED \& TRACKED}
With CMM 2 you become an adult. We are no longer negligent children. The expectations for controls are known and practices are tailored to meet those specific requirements. The performances of the base practices are planned, tracked and verified. Things are no longer done randomly. CMM 2 practices are considered "audit ready" that is, diligence and care in carrying out checks can be demonstrated and this is precisely what marks the overcoming of the negligence threshold. If we really had to make a criticism of CMM 2 practices it is that they are narrowly-focused and are not organization-wide.
\subsubsection{CMM 3 – WELL DEFINED}
The next step is marked by well-defined and standardized practices across the organization. Before being carried out, the practices are approved and tailored to follow standard and documented processes. It can be argued that CMM 3 practices focus on security over compliance, where compliance is a natural byproduct of those secure practices. For medium/large organizations this translates into the presence of a CISO or in any case a very competent figure in terms of cybersecurity who has the authority to manage the resources.
\subsubsection{CMM 4 – QUANTITATIVELY CONTROLLED}
The added factor in this level is the presence of metrics to enable governance oversight. This means that there are quantifiable and objective data to measure process performance. Precisely for this reason, the CMM 4 practices are also considered "audit ready". It is unrealistic for a small organization to reach this level, but, in the case of medium/large organizations, business stakeholders are made aware of the status of cybersecurity and privacy program. 
\subsubsection{CMM 5 – CONTINUOUSLY IMPROVING}
The final level. Only the best can aspire to that. Not only the practices are well-defined, standardized and detailed with objective metrics, but the process is constantly improving. Although not expressly requested, it is not uncommon to use Artificial Intelligence (AI) and Machine Learning (ML) to evaluate performance and make continuous adjustments.
\subsection{How to use}
The tools offered by the SCF are not complicated: we are talking about an Excel sheet after all. A little more complicated is how to use it and how to benefit from these CMMs. The keyword is 'planning": if you fail to plan you plan to fail! CMMs allow us to identify a target maturity level to which our organization must aspire for the next years. The CMM targets evolve each year. Without maturity goals, it is very difficult and subjective to define success for a security and privacy program. CMMs also help us understand how we want our work environment to be. The transition from CMM 3 to CMM 4 (or from 4 to 5), for example, needs to be studied carefully. CMM 3 maturity is generally considered “an environment that is in control” whereas being in a CMM 4 environment is more of a “controlled environment” that is more controlled and less free. The cost to mature from a CMM 3 to a CMM 4 could be hundreds of thousands to millions of dollars, so there is a very real cost associated with picking a target maturity level. From a slightly more selfish point of view, the maturity level allows the CISO to demonstrate a defined resourcing need in the case of his request is denied, thus maintaining a clean conscience. We anticipated the so-called "sweet spot". A mentally stable person who wants to plan the resources required for a project knows that at least he will have to meet the requirements of CMM 2 and that most likely (except in exceptional cases) CMM 4 will suffice. The maturity levels, in fact, in addition to setting objectives, also set minimum starting requirements. Last, but not least: the maturity levels also characterize the types of third-party services we can rely on.
\section{Frameworks compared}
At this point you might think: "Wow! So many frameworks, amazing! Which one should I choose?". You will find the answer to this question within yourself, my little friend. There is no framework that is better than the others, but only the one that best suits your needs. Furthermore, nobody forbids you to start with one and continue with another at a later time. The next chart can help you decide which framework is right for you.
\begin{figure}[H]
  \centering
  \includesvg[inkscapelatex=false,width=\textwidth]{heatmap.svg}
  \caption{Frameworks comparison}
\end{figure}
\noindent
Let's take our dear farmer as an example, we had left him in his flooded field. Seeing his beloved cucumbers destroyed, has opened his eyes and mind. Since that day he has decided to change and make his small cucumber field a big agricultural industry. This drastic change forces it to submit to many more regulations. It will not process sensitive data and therefore a framework focused on privacy can be put aside, but electronic controls will always be present to manage the entire supply chain and therefore one of the remaining ones will certainly be applied. Returning to the frameworks we have analyzed, let's try to highlight the peculiarities of each of them. The first big distinction we can make is between opensource and paid frameworks (such as ISO certification). Cost, I repeat, is a very important discriminating factor within an organization: if we have unlimited money, we could simply rely on insurance for any problem (not taking into account the damage to reputation). To the simple cost of certification we must add the cost of the resources necessary for implementation. For this reason, it is also important to evaluate the level of automation of the framework which can reduce the cost of working hours.\newline
The type of organization determines the obligations to which it must comply. The NIST CSF, for example, is focused on safeguarding critical infrastructures (albeit adaptable to many others) and its controls could be too invasive for a small reality such as that of our farmer friend. In the same way, a framework such as the SCF, in my opinion, can be too complicated for a neophyte because it is more dispersive and less structured. A second distinction that can be made between the numerous frameworks (analyzed in this work and not) is, in fact, on the basis of their "specialization". The numerous controls of the SCF allow it to adapt to any organization, be it critical or not, but for this reason a certain familiarity with cybersecurity concepts is required to implement them. A framework such as the National Framework for Cybersecurity and Data Protection (they should find an acronym), is a cross between the adaptability of the SCF and the specificity of the NIST CSF. When we analized it, we highlighted the presence of the so-called "Essential Cybersecurity Checks" which allow even small organizations to enter the world of cybersecurity.\newline
Third distinction: "rigor". Some frameworks, mainly those with certification such as ISO 27005 or specific to a sector, are more stringent on the controls to be performed and leave less freedom to the organization that implements them.\newline
A final distinction concerns the "breadth of coverage". The controls of the NIST CSF or the...National Framework for Cybersecurity and Data Protection, are more like "recommendations" compared to the controls and procedures of the ISO 27005 or the SCF. 