\chapter*{Introduction}
\addcontentsline{toc}{chapter}{Introduction}
In the last few years, Cyber Risks Management became a core practice inside any organization which deal with full or partially digital assets (intended as all those resources that depend in some way on a computer). Or so it should be. The presence of smart devices and ever-increasing connectivity have led organizations to work within cyberspace with its benefits and dangers. Not everyone is ready for this journey and some will succumb along the way. And this is where I come into play. My task is to provide you with a guide that can benefit everyone, regardless of their skill level, from beginners to experts, from dummies to masters, and which lays the foundations for the research towards the much desired cybersecurity. The answer to all your questions will not be "42", but something a little bit more complicated. This work is presented as a guide and is structured as such, but don't expect instructions like those of "Ikea" fornitures. Numerous approaches to cyber risk management will be shown and analyzed. What steps must be followed for a correct use of cyberspace? We will have at our disposal systematic methodologies to detect and reduce vulnerabilities and threats (e.g. MEHARI, OWASP etc.) or more complex and structured frameworks such as the Cyber Security Framework provided by NIST. There will be dummy-proof examples (I tried very hard to get to that level) and colorful figures. The final focus of this survey will be cyber risk management within critical infrastructures, in particular healthcare infrastructure. In the last period, in fact, healthcare has been the victim of numerous cyber attacks and my goal is to explain why and how to prevent this from happening again.\newline
The rest of the survey is organized as follows. Chapter 1 introduces the concept of risk management within an organization (but outside the cyberspace) by examining ISO-31000 risk management framework. Chapter 2 shifts the focus to the cyber component of risk management, briefly listing the new opportunities and problems that cyberspace offers us. Taking into consideration the benefits and dangers of cyberspace, chapter 3 analyzes the main "weapons" available to those we will call guardians of the cyberspace, chapter 4 lists the main methodologies to face systematic cyber threats and chapter 5 details some of the major cyber risk management frameworks and their similarities and differences. These three chapters are strongly connected to each other: CRM methodologies make use of different tools and frameworks can adapt them to fit the organization's mission. Chapter 6 evaluates the current situation in healthcare infrastructure in the field of cybersecurity, proposes some solutions to the problems highlighted and a "dummy" example of what is the process to implement cyber risk management controls inside a hospital. Chapter 7 draws the conclusions. Let the journey begin.
