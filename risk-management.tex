\chapter{Risk Management}
The main purpose of this guide is to discuss how modern enterprises should manage cyber-risks while interacting inside critical environments. But before entering in details of all of this we have to understand what is risk management in general. "What is a risk ?", "Why does everyone talk about risk management ?", "I am a farmer: is it really important for my cucumber farm ?", "How do I implement it inside my organization ?". These and other questions will be answered in details in the following sections.
\section{What is it?}
History of risk management originates in the Tigris-Euphrates valley about 3200 B.C. (exactly like everything we studied at school). There lived a group called the Asipu and one of their main function was to serve as consultant for uncertain or difficult decisions: they would identify the important dimensions of the problem, the alternatives and collect data on the outcomes of each of them. After the analysis was completed, they would recommend the most favorable alternative issuing a final report to the client in a not at all uncomfortable clay tablet. The bad part is that the best available data were divine signs, not easily quantifiable, and if the prediction was faulty they could easily blame a playful God. Fortunately things changed a bit through the years and human started using more easy to read signs such as probability, even though the old techniques for mitigating or reducing risks are still very effective and used \cite{Covello1985}.
Back to the present, every business or enterprise, from the small corner store to the large manufacturer, faces risks everyday: factory buildings can be damaged by fire slowing the supply chain, an employee could slip, fall and then sue the manager costing money to the company, financial losses can occur as a result of defective products or data breaches can damage the reputation of a data controller after the violation of the privacy of its users.
All of these are simple examples of risk, which may span across every kind of sector (not only cyber-related).
\definition{Risk}{\begin{center}"Effect of uncertainty on objectives" \cite{ISO2018}\medskip\end{center}  Practically speaking, a risk is "the possibility of something bad happening", in every possible context an organization may operate. A risk can be objective or quantifiable such as a financial loss or subjective and difficult to quantify such as a reputation loss.}Risk management, as you may understand by yourself, aims to manage every risk situations inside an organization. More technically is a continuous, gradual and proactive process that aim to "direct and control the organization with regard to risk" \cite{ISO2018}: in few words, reduce the likelihood of risks to prevent disasters. Fundamental purpose of enterprise risk management, anyway, is not to just protect, but enhance and create value for the organization. An organization recognized as safe will certainly be preferred to one with serious security flaws and therefore will have greater opportunities for investment or contractual proposals.
\section{Is it important?}
Of course it is, otherwise we wouldn't be here talking about it. The goal of every successful business is to gain value and reduce losses (both in terms of money, reputation and so on) and risk management is the dividing line between these two scenarios. Focusing the attention on risk, a business can (try to) protect its assets (those things on which protection is worth investing) before it's too late. The saying "prevention is better than cure" also applies in this context: investing in prevention certainly has a cost, but in most cases it is lower than the cost of treatment in case of disaster (and reduces the likelihood of such disaster). When a potential risk has been identified, mitigate it is (most of the times) easy or at least doable. Managing the after effects of a risk, instead, is harder and not always possible: is it easier and more convenient to prevent a fire putting in place a working fire extinguishing system or to rebuild the entire burned-out building?\newline
Going back to the farmer question: YES! Also a cucumber plantation has risks (flooding, heating, parasites and other environmental risks) and a good farmer has to manage them in a proper way, for example using pesticides to prevent parasites that would destroy the plantation (other cucumber risks are not covered in this paper).
We can use the farmer as an example to introduce other important element of risk management, the stakeholder.
\definition{Stakeholder}{\begin{center}"Person or organization that can affect, be affected by, or perceive themselves to be affected by a decision or activity." \cite{ISO2018}\medskip\end{center} Simply the ones that will pay the consequences of anything that was decided regarding risk: if a flood destroy the culture, the farmer won't be able to sell cucumbers at Sunday market, losing money and clients. The farmer is the stakeholder: his decision to not invest in "anti-flood" measures led him to no longer have anything to sell.}
\section{How to implement it?}
The term Risk management refers to the set of processes through which an organization identifies, analyzes, quantifies, eliminates and monitors the risks associated with a specific production process. We stated before that risk management is a continuous, gradual and proactive process that aim to reduce risks or at least the damage caused by them. Gradual because is subdivided in multiple subsequent steps with different features and goals. Proactive because focuses mainly on prevention from incidents, reducing risks' likelihood, and not on the recovery from the derived problems (that is anyway a crucial part of every business: every organization should have a B plan). Continuous because is repeated in loop, adding at each iteration additional information acquired from previous one and improving itself. But! Here rises a problem. The process we are talking about is not properly step-by-step and there isn't a guide to follow with the exact actions to take inside each organization. Different organizations have to satisfy different requirements (also imposed by government) according to own area of expertise, size or geographic location, and it is not feasible to write a different process for each of them. For this reason someone decided to write down the guidelines and best practices that should (NOT must) be followed for an healthy risk management plan. Some of the brightest minds on earth grouped together to draw up what are called \textit{standard}.\definition{Standard}{\begin{center}A published statement on a topic specifying characteristics, usually measurable, that must be satisfied or achieved in order to comply with the standard." \cite{Paulsen2019}\medskip\end{center} Risk management standards are like a guide to help ensuring that risk management is carried out in a proper way. Standards usually include checkpoints and examples, to make it really easy for organizations to comply with them (exactly like this guide, maybe it will be converted in standard in the near future).}"ISO 31000" by \textit{International Organization for Standardization} (ISO) and "Enterprise Risk Management" by \textit{The Committee of Sponsoring Organizations of the Treadway Commission} (COSO) are the two leading risk management standards in the world today. Even though they are very similar (trivially they have the same goal) differences between them far outnumber similarities. COSO ERM is targeted more toward people in accounting and audit while ISO 31000 is written for anyone interested in risk management. ISO 31000 is few pages long while COSO ERM needs you to take holidays to finish it. The choice between one or the other consists on which one fits the organization and the good part of this game is that is possible to choose all of them (or others) and take the best of each. In the following section we will see an overview of ISO 31000 to get an idea of how these standards are structured, but going on with this guide we will focus on more specific standards for cyber-related risk and their methodologies.

\section{Standard example: ISO 31000}
I decided to use ISO 31000 \cite{ISO2018} as example, for few basic reasons: is concise, easy to follow, can be read in less than an hour and can be applied to pretty much any industry, culture, and language. The standard consists of 3 main components: Principles, Framework and Process. Let's deepen it together. 
\subsection{Principles}
The principles are the characteristics that an effective and efficient risk management should have, communicating its value and explaining its intention and purpose. The principles are the core element for managing risk and should be considered when establishing the organization’s risk management framework and processes. These principles should enable an organization to manage the effects of uncertainty on its objectives. Consider them the "eight" commandments.
\begin{figure}[H]
  \centering
  \includesvg[inkscapelatex=false,width=0.5\textwidth]{iso-1.svg}
  \caption{Principles}
\end{figure}
\noindent
\begin{itemize}
    \item \textbf{Integrated:} Risk management is a fundamental part of all activities inside an organization.
    \item \textbf{Structured and comprehensive:} The approach to risk management should be structured and comprehensive in order to have consistent and comparable results. It's not a free-for-all!
    \item \textbf{Customized:} Risk management framework and process can be customized by each organization to fit their own context and objectives. 
    \item \textbf{Inclusive:} Is important to involve stakeholders in risk management plan and consider their knowledge, view and perceptions while making decisions. This results in improved awareness and informed risk management.
    \item \textbf{Dynamic:} Organization's internal and external context can change rapidly and risk can emerge, change or disappear just as quickly. Risk management must anticipates, detects, acknowledges and responds to those changes as soon as possible.
    \item \textbf{Best available information:} Historical and current information, as well as future expectations are the inputs of risk management. Risk management takes into account any limitations and uncertainties associated with such information and expectations. Information should be timely, clear and available to relevant stakeholders.
    \item \textbf{Human and cultural factors:} Risk management is significantly influenced by human behaviour and culture at each level and stage.
    \item \textbf{Continual improvement:} Risk management is a loop that improve itself at each iteration.
\end{itemize}
\subsection{Framework}
A Risk management framework assists organization in integrating risk management into its main activities and functions. Its effectiveness will depend on its degree of integration into the governance of the organization, that is the system by which an organization makes and implements decisions in pursuit of its objectives. The more or less pyramidal system in which the boss at the top gives orders to the subordinates of the lower levels, but for which he has the responsibility and on which the fate of the organization depends. Fundamental is the support from stakeholders, in particular from top management. Its features, modeled around Leadership and Commitment, are subdivided into Integration, Design, Implementation, Evaluation and Improvement, taking into account the fact that the way in which they work together should be customized to the needs of organization.
\begin{figure}[H]
  \centering
  \includesvg[inkscapelatex=false,width=0.5\textwidth]{iso-2.svg}
  \caption{Framework}
\end{figure}
\noindent
\subsubsection{Leadership and commitment}
In risk management framework, everything rotates around top management and oversight bodies. They should ensure (if they don't want to be fired) that risk management is well integrated into all organizational activities, demonstrating leadership and commitment. They are required to: customize and implement inside the organization all components of the framework, issue a set of rule or guidelines for the desired risk management approach, ensure the availability of necessary resources, subdivide the authority, responsibility and accountability at appropriate levels within the organization.
All of this will help the organization to align risk management with its objectives and strategy, recognize and address all obligations and voluntary commitments, establish which kind of risk (and how many) the organization should take, inform the stakeholders about them and the importance of their management, promote monitoring and ensure that the framework remains coherent with the context of the organization.\newline
Oversight bodies answer the question "Who controls the controller?" If the top management is required to manage risk, the oversight bodies are accountable for overseeing risk management. Their main tasks are: ensure that risks are taken into account during organization's objectives setting, understand the possible risk that may rise while facing these objectives, ensure that the managing system is well implemented and operative and that every important information about risk and their management is communicated in the proper way.
\subsubsection{Integration}
"To conquer fear, you must become fear" said Ra's al Ghul to Bruce Wayne in the process of transformation into Batman. Same story to integrate risk management into organizational structure. I'm not asking you to become the structure of your organization anyway, just to understand it. Structures differ depending on the organization's purpose, goals and complexity and risks may rise in every part of them, giving a portion of responsibility for managing risk to everyone. Risk management should be part of the behaviour of the organization (purpose, governance, strategy and objectives) and not a separated entity.
\subsubsection{Design}
First things first, understand the organization and its context, both internal (e.g. vision, mission and values of organization) and external (e.g. geographic area and external relationships).\newline
Top management and oversight bodies should demonstrate and articulate their commitment to risk management through a policy or whatever form of document that clearly expresses organization's objectives and commitment to risk management. This should be communicated within organization and to stakeholders.\newline
A correct risk management framework design requires a proper subdivision of roles (authorities, responsibilities and accountabilities) and allocation of appropriate resources (people, skills, documents, training).\newline
Last but not least: communication and consultation. Communication means sharing information with targeted audiences. Consultation means listening to feedback to shape decisions or other activities.
\subsubsection{Implementation}
Previous steps are important, but now is time to make them effective. Organization should develop an appropriate plan including time and resources, identify where, when, how and by whom decisions are made modifying the decision-making processes if needed and ensure that all necessary arrangements are well understood and practised. Like always, stakeholders should be engaged and aware to take into account any new or subsequent uncertainty as it arises.\newline
Design and Implementation will ensure that risk management is part of all activities within the organization.
\subsubsection{Evaluation}
The organization should periodically measure the framework performance against expected results and determine whether it is still efficient to support achieving the objectives.
\subsubsection{Improvement}
Follows the "Improvise. Adapt. Overcome" Bear Grylls's mentality, without the part of improvisation. Organization should monitor internal and external changes to adapt the risk management framework to them. Continually improving the suitability, adequacy and effectiveness of the risk management framework contributes to the enhancement of risk management.
\subsection{Process}
As said before, the risk management process is an iterative loop that like previous elements should be integrated in into the structure, operations and processes of the organization. Inside an organization there can be many applications of risk management process, customized to achieve the objectives.
\begin{figure}[H]
  \centering
  \includesvg[inkscapelatex=false,width=0.5\textwidth]{iso-3.svg}
  \caption{Process}
\end{figure}
\noindent
\subsubsection{Communication and consultation}
Stakeholders have to understand risk, know the basis on which decisions are made and why particular actions are required. The purpose of Communication and consultation is just that: the first aims to promote awareness providing, for example, sufficient information to facilitate risk oversight and decision-making, the latter involves obtaining feedback and supporting decision-making ensuring that different views are appropriately considered during the process. Both internal and external stakeholders should be part of the communication and consultation step.
\subsubsection{Scope, context and criteria}
To adapt the process to the organization's profile is necessary to establish the scope, the context and criteria.\newline
The scope regards the organizational objectives that must be taken into account at all different levels of the risk management process (e.g. strategic, operational, programme, project, or other activities). Is important to consider objectives and decisions that need to be made, the expected outcomes from each step, required resources, tools and techniques and relationship with other projects or activities.\newline
Context, internal or external, is the environment in which the organization seeks to define and achieve its objectives and where the risk management process is applied. Understanding the context is fundamental because, trivially, is where the risk may rise (at least the one the organization is interested in).\newline
Criteria are used to evaluate the type of risk that organization may or may not take, their significance and to support decision-making process. They should reflect the organization's values, objectives and resources: chip shortage is a risk that our farmer friend can take without problems. Criteria should be established at the beginning of the risk management process, but being dynamic they should be continually reviewed.
\subsubsection{Risk assessment}
Risk assessment consists of risk identification, analysis and evaluation. It should be conducted systematically, iteratively and collaboratively and should use the best available information, supplemented by further enquiry as necessary.\newline
Identification means find, recognize and describe the risks that may disturb organization's path to the success. The goal is to identify uncertainties (whether or not their sources are under organization's control) that may effect one or more objectives.\newline
Risk analysis aims to understand the nature of risk and its level. Given the fact that we are talking about uncertainty, the more information about risk are available the easier will be to comprehend it: sources, consequences, likelihood, events, scenarios, controls and their effectiveness. The analysis complexity may change according to its purpose or availability of information, can be qualitative, quantitative or a combination of these depending on circumstances.\newline
Risk evaluation supports decisions: compares results of risk analysis with risk criteria to determine where additional action is required (e.g. reconsider objectives or further analysis).
\subsubsection{Risk treatment}
Self explanatory: how should we handle the risk? An iterative (again) process that consists of: formulating and selecting risk treatment options, planning and implementing them, assessing their effectiveness, deciding whether the remaining risk is acceptable or not (in this case more treatment needed).\newline
To select the best treatment option(s) is necessary to balance potential benefits with costs, effort and disadvantages of implementation. Common options are risk sharing (e.g. buying insurance), change likelihood or consequences, remove sources, take more risk or completely avoid the activity that generate the risk. Note that the criteria used to choose the right option is not solely economic, but take into account all organization's obligations: it would be a lot easier, for the farmer, to spread an huge amount of insecticide to protect his cultivation, but this may violate the limits imposed by law.\newline
When the best option is chosen, is time to prepare and implement the risk treatment plan: what is the implementation order of the treatment? Who is accountable and responsible for implementation? Which resources are required? What are the actions to be taken? When?
\subsubsection{Monitoring and review}
An iterative process wouldn't have made sense without its constant monitoring. Ongoing monitoring and periodic review assure and improve the quality and effectiveness of process design and should take place in all stages of it, with clearly defined responsibilities.
\subsubsection{Recording and reporting}
All information resulting from the risk management process should be recorded and reported through appropriate mechanisms. Reporting is fundamental in the organization’s governance and should enhance the quality of dialogue with stakeholders and support top management and oversight bodies in meeting their responsibilities.\vspace{3cm}\newline
This wall of text may have been difficult to digest. And that's just the beginning! But let's try to make sense of it. The framework examined is only one of the many used by organizations, some features may (and will) differ from those of other standards and no one will ever ask you to follow every step of any framework you want to implement to the letter. The framework acts as a backbone, as a model to follow, but it is not a checklist to complete. The "suggested" actions (you must however follow them wisely) have the aim of increasing the synergy between the technologies in use, the processes in which they are used and the people who use them. It is important to grasp all the relationships between the components (human or not) within the organization and the framework provides guidelines to do this in the best possible way.

